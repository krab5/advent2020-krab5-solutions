\documentclass{article}

\usepackage[T1]{fontenc}
\usepackage[utf8]{inputenc}
\usepackage[english]{babel}
\usepackage{amsmath}
\usepackage{amssymb}
\usepackage{amsthm}
\usepackage{subcaption}
\usepackage{listings}
\usepackage[hmargin=1.0in,vmargin=1.0in]{geometry}

\newtheorem{definition}{Definition}
\newtheorem{theorem}{Theorem}
\newtheorem{property}{Property}
\newtheorem{lemma}{Lemma}

\newcommand{\divi}{\mathbin{\mid}}
\newcommand{\tdiv}{\mathbin{~\mathrm{div}~}}
\newcommand{\tmod}{\mathbin{~\mathrm{mod}~}}

\title{Challenge Day 13 -- Arithmetics and Buses}
\author{}
\date{}

\begin{document}
\maketitle

Let $I \subseteq \mathbb{N}$ be a finite, indexing set, starting at 0\footnote{$I$ does not have to be finite for the reasoning, but it does for the algorithm. Also, it technically does not have to start with 0, although it is more convenient (one can always obtain $I' \subseteq \mathbb{N}$ starting at 0 from any $I$ by offsetting every $i \in I$ by $\min(I)$)}. Let $(n_i)_{i\in I}$ be a sequence of \textit{bus IDs}. Buses all start their round at the same timestamp $t = 0$ (in minutes). They complete one round in a number of minutes equal to their idea (i.e., they arrive at their beginning point at every $t = n_i * k$, $k \in \mathbb{N}$).

\medskip

\noindent\textbf{Question:} what is the earliest (i.e. smallest) timestamp $t$ such that, for all $i$, bus number $i$ arrives at $t + i$?


\section{Modelling the Problem}

For any bus ID $n_i$ ($i \in I$), we give the sequence of timestamps at which the bus arrives: $(T(n_i)_k)_{k \in \mathbb{N}}$, with:
$$\forall k \in \mathbb{N}, T(n_i)_k = k n_i$$

\paragraph{For 2 Buses} Let us say $I = \{0, i_1\}$ for the sake of explanation. We want to find the earliest $\hat t$ such that:
$$\exists \hat k_0, \hat k_1 \in \mathbb{N}, \hat t = T(n_0)_{\hat k_0} \ \mathrm{and}\ T(n_0)_{\hat k_0} = T(n_{i_1})_{\hat k_1} + i_1$$

To do so, we first try to fin any $k_0, k_1$ such that:
\begin{equation}\label{eqn:2buses}
k_0 n_0 - k_1 n_{i_1} = i_1
\end{equation}

\paragraph{For $N$ Buses ($N \geq 2$)} For any number of buses, we obtain an $N-1$ equations, $N$ unknowns system:
\begin{equation}\label{eqn:nbuses}
\left\{\begin{array}{lcl}
k_0 n_0 - k_1 n_{i_1} & = & i_1 \\
k_0 n_0 - k_2 n_{i_2} & = & i_2 \\
& \vdots & \\
k_0 n_0 - k_N n_{i_N} & = & i_N
\end{array}\right.
\end{equation}

Note that any equation ($i \in I \setminus \{ 0 \}$) of this system generate a (countable) infinite family of solutions (if it is solvable; hopefully it is) $\{ (k_0,k_i) \}$. The intersection of these families gives another ``sparser'', yet infinite, family of solutions for $k_0$. The positive minimum of this family, $\hat k_0$, allows us to calculate the answer to the question, $\hat t = \hat k_0 n_0$.


\section{Mathematical Background}

In this section, we provide a bit of background in arithmetics as to help understanding the resolution of the problem.

Arithmetics is the branch of mathematics that handles integers. We denote $\mathbb{Z}$ the set of integers (or ``relative'' integers), i.e. numbers that are not decimal or reals, and may be negative or positive. $\mathbb{N}$ is the subset of $\mathbb{Z}$ consisting of positive integers (also called \textit{natural numbers}).

\subsection{Divisibility, Greatest Common Divider}

One of the first major concept of arithmetics is the concept of \textit{divisibility}. Indeed, in the domain of integers, dividing two numbers does not always have sense. For instance, 6 divided by 2 is 3, which is okay; but 3 divided by 2 is not an integer number (it ``does not exist'' in the realms of integers).

\begin{definition}[Divisibility]
Let $a,b \in \mathbb{Z}$. We say (equivalently) that:
\begin{itemize}
\item $a$ is \textbf{divisible by} $b$
\item $b$ is a \textbf{divider} of $a$
\item $a$ is a \textbf{multiple} of $b$
\end{itemize}
\noindent and we write $a \divi b$ if there exists $k \in \mathbb{Z}$ such that $a = k b$.
\end{definition}

Note that 0 is divisible by everything, but nothing is divisible by 0 (as for any $k$, $0 k = 0$). Note also that everything is divisible by 1 but 1 is divisible by nothing else than itself (as $k = 1 k$). Finally, note that $a \divi a$ for any $a \in \mathbb{Z} \setminus \{ 0 \}$ (because $a = 1 a$).

If $a \divi b$ then $-a \divi b$ (i.e. if $a = k b$ then $- a = (- k) b$) and, if $a \divi b$ then $a \divi -b$ (i.e. if $a = k b$ then $a = (-k) (-b)$); this means that, in general, handling natural numbers is "mostly the same" as handling integers.

\begin{definition}[Set of Dividers]
Let $a \in \mathbb{Z}$. We note $\Delta(a)$ the set of (positive) dividers of $a$. Formally:
$$\Delta(a) = \{ b \in \mathbb{N} \mid a \divi b \}$$
\end{definition}

Note that we could consider the set of \textbf{all} dividers of $a$, which $\Delta(a)$ plus the opposite of any number in $\Delta(a)$. Following our priori remark though, this is unnecessary.

\begin{property}[Properties of the Set of Dividers]
Let $a \in \mathbb{Z}$. The set $\Delta(a)$ has the following properties:
\begin{itemize}
\item $a \in \Delta(a)$ and $1 \in \Delta(a)$ (because $a = 1 \times a$) ;
\item $\forall b \in \Delta(a), 0 < b \leq a$ ;
\item $\Delta(0) = \mathbb{N} \setminus \{ 0 \}$, $\Delta(1) = \{ 1 \}$ ;
\end{itemize}
\end{property}

The set of dividers is convenient to formally define \textit{prime numbers}.

\begin{definition}[Prime Number]
Let $p \in \mathbb{N}$. $p$ is \textbf{prime} if and only if the cardinality of its set of dividers is exactly equal to $2$:
$$p \ \mathrm{prime} \ \Leftrightarrow\ \mathrm{card}(\Delta(p)) = 2$$
\end{definition}

From this definition, and by noticing that we always have $a \in \Delta(p)$ and $1 \in \Delta(p)$, we note that this definition is equivalent to ``\textit{$p \in \mathbb{N}$ is prime if and only if its only dividers is $1$ and itself.}''

In particular, this means that 0 and 1 are \textbf{not} prime numbers.

\medskip

Finally, let us take $a,b \in \mathbb{Z}$. We note $\Delta(a,b) = \Delta(a) \cap \Delta(b)$ the intersection of the set of dividers of $a$ and $b$. If $a$ and $b$ are not null, this intersection is never empty ($1 \in \Delta(a)$ and $1 \in \Delta(b)$).

By the definition of the intersection, $\Delta(a,b)$ contains any number that divides \textbf{both} $a$ \textbf{and} $b$. This set is usually called \textit{set of the common dividers of $a$ and $b$}.

Because $\Delta(a)$ and $\Delta(b)$ are (lower and upper) bounded subsets of $\mathbb{Z}$, they are finite. Subsequently, $\Delta(a,b)$ is finite as well, and it thus admits a maximum:

\begin{definition}[Greatest Common Divider (GCD)]
Let $a, b \in \mathbb{Z}$ not identically null (i.e. $a$ and $b$ cannot be simultaneously null). The maximum of the set set of common dividers $\max(\Delta(a,b))$ is called the \textbf{greatest common divider} of $a$ and $b$.

It is usually denoted $\mathrm{gcd}(a,b)$, $\delta(a,b)$ or $a \wedge b$.
\end{definition}

If $a$ and $b$ are not identically null, their GCD always exists, and is necessarily greater or equal to 1. If it is equal to 1, we say that $a$ and $b$ are co-prime:

\begin{definition}[Co-prime Numbers]
Let $a,b \in \mathbb{Z}$ not identically null. $a$ and $b$ are \textbf{co-prime} if and only if their GCD is equal to 1:
$$\delta(a,b) = 1$$
\end{definition}

The GCD has some interesting properties, that directly come from the nature of divisibility:

\begin{property}[Some Properties of the GCD]
Let $a,b \in \mathbb{Z}$ not identically null:
\begin{enumerate}
\item $a \divi \delta(a,b)$ and $b \divi \delta(a,b)$ (by definition) ;
\item $\delta(a,b) = \delta(b,a)$ (because intersection is commutative) ;
\item $0 < \delta(a,b) < \min(a,b)$ ;
\item $\delta(1,a) = \delta(a,1) = 1$ (consequence of the previous property) ;
\item if $a \neq 0$, $\delta(a,a) = a$ ;
\item $\delta(-a,-b) = \delta(-a,b) = \delta(a,-b) = \delta(a,b)$ ;
\item if $a \neq 0$, $\delta(a,0) = a$ (because $\Delta(0) = \mathbb{N} \setminus \{ 0 \}$) ;
\item for any $c \mathbf{Z} \setminus \{ 0 \}$: $\delta(a c, b c) = \delta(a,b)$ ;
\end{enumerate}
\end{property}

A quick note on the case where $a = b = 0$: it is tempting to write $\delta(0,0) = 0$ because of the fifth and seventh properties presented here. Formally, this is wrong: \textbf{the GCD is \textit{not} defined for $(0,0)$}. The reason for that is that $\Delta(0) = \mathbb{N}$, and thus that $\Delta(0,0) = \mathbb{N} \cap \mathbb{N} = \mathbb{N}$. $\mathbb{N}$ being unbounded, it \textbf{does not have a maximum}.

The formula $\delta(0,0) = \max(\Delta(0,0))$ is thus \textbf{invalid and meaningless}.

Some people perform some kind of \textit{function prolongation}, with the argument that ``we deduce naturally from properties 5 and 7 that $\delta(0,0)$ must surely equal to $0$''. You do that if you want, but this is technically wrong, and may cause errors in demonstrations. Be careful!


\subsection{Euclidean Division and Euclid Algorithm}

Rational and real numbers have a \textit{division operator}, generally denoted $\cdot / \cdot$. Given two number $a$ and $b$ with $b$ not null, $q = a / b$ is a number such that $a = b q$.

This division operation, as it is formulated right now, does not work for any pair of integers. In fact, it only work for pairs $(a,b) \in \mathbb{Z}^2$ such that $b \neq 0$ and $a \divi b$. By definition, $a \divi b$ implies that there exists $k \in \mathbb{Z}$ such that $a = k b$. We see here that, if $a$ is divisible by $b$, then $a / b$ is equal to that $k$.

For any pair $(a,b)$ such that $a \not\divi b$ ($a$ not divisible by $b$), the $/$ operator is \textbf{not defined}; i.e. $a / b$ does not exist.

\medskip

That being said, by twisting a little bit the concept of division, we can create a new type of division that works on any number (except 0): the Euclidean division.

\begin{definition}[Euclidean Division]
Let $a,b \in \mathbb{Z}$ and $b \neq 0$. There exist $q \in \mathbb{Z}$ and $r \in \mathbb{N}$ such that:
$$a = b q + r$$
\noindent with $0 \leq r < b$. The pair $(q,r)$ is the result of the \textbf{Euclidean division} of $a$ by $b$. $q$ is called the \textbf{quotient}, and $r$ is called the \textbf{remainder}.
\end{definition}

In general, the quotient of the Euclidean division of $a$ by $b$ is denoted $\div$ or $\tdiv$. The remainder is usually denoted $\tmod$, or $\%$. The Euclidean division is usually what people mean when they think of ``integer'' division.

Euclidean division has numerous interesting properties.

\begin{property}[Properties of the Euclidean Division]
Let $a,b \in \mathbb{Z}$ and $b \neq 0$:
\begin{enumerate}
\item $a \divi b \Leftrightarrow a \tmod b = 0$, and $a \div b = a / b$ ;
\item $a = a 1 + 0$, so $a \div 1 = a$ and $a \tmod 1 = 0$ ;
\item $1 = 0 b + 1$, so $1 \div b = 0$ and $1 \tmod b = 1$ ;
\item More generally, if $b > a$: $a = 0 b + a$, so $a \div b = 0$ and $a \tmod b = a$ ;
\end{enumerate}
\end{property}

There is an interesting connection between GCD and Euclidean divisions, that leads to the definition of the Euclid algorithm:

\begin{theorem}[Euclidean Division and GCD]
Let $a,b \in \mathbb{Z}$ with $b \neq 0$. Let $(q,r)$ be the result of the Euclidean division of $a$ by $b$ (i.e. $q = a \div b$ and $r = a \tmod b$). We have:
$$\delta(a,b) = \delta(b,r)$$
\end{theorem}

Finding the GCD of $a$ and $b$ is thus like calculating $r = a \tmod b$ and finding the GCD of $b$ and $r$, which can be obtained the same way (i.e. $r' = b \tmod r$ and finding the GCD of $r$ and $r'$, and so on) until $r = 0$ (because then $\delta(x,0) = x$).

This procedure is called \textit{Euclid's algorithm}. Note that it is recursive.



\subsection{Diophantine Equations}

We recall the concept of Diophantine equations, and interest ourselves to the particular case of linear, two-unknown equations, close to our problem.

\begin{definition}[Diophantine Equation]
A Diophantine equation is any polynomial equation that admits integer solutions.
\end{definition}

\begin{definition}[Linear Diophantine Equation]
A \textbf{linear} Diophantine equation is a Diophantine equation where the unknowns are monomic, i.e. elevated to the power 1. Formally:
$$a_1 x_1 + a_2 x_2 + \ldots + a_N x_N = c$$

\noindent where $(a_k)$ and $c$ are constants and $(x_k)$ are the unknowns.
\end{definition}

We interest ourselves to linear Diophantine equations with two unknowns for now (i.e. $N = 2$).

\begin{definition}[Bézout Identity]
Let $a,b,c \in \mathbb{Z}$ with $a$ and $b$ not identically null. A two-unknown linear Diophantine equation
$$a x + b y = c$$
\noindent where $c$ is the \textbf{greatest common divider} of $a$ and $b$ is called a \textbf{Bézout identity}.
\end{definition}

Bézout identity are interesting because they are ``fairly easily solvable''. Note that general case of two-unknown linear Diophantine equations can be reduced to Bézout identities under certain conditions.

\begin{theorem}[Solvability and Reduction of Diophantine Equations]\label{th:solv}
Let $a,b,c \in \mathbb{Z}$ with $a$ and $b$ not identically null. If $c \divi \delta(a,b)$, then the equation:
$$a x + b y = c$$
\noindent is equivalent to the equation:
$$a' x + b' y = c'$$
\noident where $a = a' \delta(a,b)$, $b = b' \delta(a,b)$, $c = c' \delta(a,b)$. This new equation is a Bézout identity.
\end{theorem}

To resolve Bézout identities, we rely on the Bachet--Bézout theorem:

\begin{theorem}[Bachet--Bézout Theorem]
Let $a,b \in \mathbb{Z}$ with $a$ and $b$ \textbf{co-prime}. Then the equation
$$a x + b y = 1$$
\noindent always admits a (countable) infinite number of solutions. If $(x_0,y_0)$ is a particular solution of the equation, then the set of solutions to this equation is:
$$\{ (x_0 + b k, y_0 - a k) \mid k \in \mathbb{Z} \}$$
\end{theorem}

The difficulty then is to find a particular solution $(x_0,y_0)$ to the equation. One way of doing so is by using the \textit{extended Euclid algorithm} (more on that later). For now, suppose we managed to find such a particular solution.

\begin{theorem}[Bézout Identity Solutions]
Let $a,b,c \in \mathbb{Z}$ where $a$ and $b$ are \textbf{co-prime}. Let $(x_0,y_0)$ be a particular solution of the equation 
$$a x + b y = 1$$
\noindent then the equation $a x + b y = c$ admits a (countable) infinite number of solutions. A particular solution of this equation $(x_1,y_1)$ is then $(c x_0, c y_1)$, and the set of solutions is:
$$\{ (x_1 + b k, y_1 - a k) \mid k \in \mathbb{Z} \}$$
\end{theorem}

Note that, by virtue of Theorem \ref{th:solv}, the solutions of $a x + b y = c$ are the solutions of $n a x + n b y = n c$ for any $n \neq 0$.



\end{document}


